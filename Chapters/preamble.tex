\documentclass[nofonts, oneside]{ctexbook}

\usepackage{geometry}
\usepackage{fontspec}
\usepackage{xeCJK}
\usepackage{amssymb}
\usepackage{graphics,graphicx,pdfpages}
% for tikz
\usepackage{tikz}
% for varwidth
\usepackage{varwidth}
\usepackage[
hyperfigures=true,
hyperindex=true,
pdfpagelayout = SinglePage,
%pdfpagelayout = TwoPageRight,
pdfpagelabels = true,
pdfstartview = FitV,
colorlinks,
pdfborder=001,
linkcolor=black,
anchorcolor=black,
citecolor=black,
]{hyperref}
%\usepackage{fancyhdr}
\usepackage{verbatim}
% for float's caption
\usepackage{caption}
% for exercise 
\usepackage{theorem}
% for varwidth
\usepackage{varwidth}
\usepackage{tikz}
% for format of contents
\usepackage{tocloft}
% for number of footnote 
\usepackage{pifont}
% 一页结束时, 脚注编号清零
\usepackage[perpage]{footmisc}
% summary 的边框
\usepackage{mdframed}
% 插图所需的宏包
\usepackage{graphicx}
% 
\usepackage[all, pdf]{xy}
% 边框
\usepackage{mdframed}
% 双栏排版
\usepackage{multicol}
% 表格单元格内换行
\usepackage{makecell}

% 脚注编号带圈
\renewcommand\thefootnote{\ding{\numexpr171+\value{footnote}}}

% from package geometry
% 为边注加边框
\let\oldmarginpar=\marginpar
\renewcommand\marginpar[1]{%
    \oldmarginpar{\framebox{#1}}%
}
\geometry{%
    margin=1cm,
    marginparsep = 0.5cm,
    marginparwidth=1cm,
    top = 2.5cm,
    bottom = 2cm,
    outer = 2.5cm,
    inner = 2.5cm
}

% page style from package fancyhdr
%\pagestyle{fancy}
%\fancyhf{}
%\fancyhead[EL]{\thepage}
%\fancyhead[ER]{\nouppercase{\leftmark}}
%\fancyhead[OR]{\thepage}
%\fancyhead[OL]{\nouppercase{\rightmark}}

% from package hyperref
\hypersetup{
	bookmarksnumbered = true,
	pdftitle = {The AWK Programming Language},
	pdfcreator = {https://m-mono.github.io},
	pdfauthor = {Alfred V. Aho, Brian W. Kernighan, Peter J. Weinberger},
	pdfsubject = {AWK程序设计语言},
	colorlinks = false,
	pdfborder = 0 0 0,
	pdfkeywords = {我认为这本书是学习AWK的最好书籍,语言发明人写的,肯定不同寻常。因为他们对AWK是最了解的,所以简明扼要却不乏深入。我们从中可以读到他们为什么发明AWK,AWK的长处和短处,AWK的简单发展史等。这本书对AWK的编程模型、基本语法有简单明了的介绍,在进行数据处理、文本处理、报表、试验算法方面的应用也有很多好的实例。
		由于是1988年的老书,其中对GAWK最新版本对AWK的扩展没有提及,但这并不妨碍其称为一本经典。
		和The C Programming Language类似,翻翻这本书的目录,你会发现,它只用了60页两个章节的篇幅介绍AWK的语法,而剩下的篇幅都是用AWK做为一个简洁紧凑的玩具语言,通过各种通俗易懂的程序,来向你展示从关系型数据库到编译器以及Unix系统上各种常用程序实现的基本原理。凭借几位大牛的深厚功力,这些内容的讲解真是深入浅出、举重若轻,让人大呼过瘾。如果你轻信了书中前言所说的,学AWK,只要看了前两章就可以了,那么你的损失可就大了。
		此外,由于AWK的语法设计上和C保持了一致,你应该可以从书中的AWK程序实例中见到很多熟悉的C的编码范式(Coding Idiom)。
		倒数第二章,作者利用AWK实现了三种排序算法和两种图遍历算法,对于大多数读者而言,可能这一章的内容会显得更加亲切一些。
		最后一章,作者介绍了AWK从最初的版本到书中介绍的版本之间的发展历程,简单说来就是,作者们最初也没想到AWK还可以做这么多事情(比如写关系型数据库和编译器)。在这一章中,作者介绍了一下AWK中Function定义里那个诡异的Local Variable的声明方式的设计由来。}
}

% from package fontspec and xeCJK
\setCJKmainfont[Path=Fonts/]{SourceHanSansCN-Normal}
\setCJKsansfont[Path=Fonts/]{SourceHanSansCN-Normal}
\setCJKmonofont[Path=Fonts/,Scale=0.9]{SourceCodePro-ExtraLight}
\setmainfont[Path=Fonts/]{SourceHanSansCN-Normal}
\setsansfont[Path=Fonts/]{SourceHanSansCN-Normal}
% "Mapping={}" make quote symbol straight
\setmonofont[Path=Fonts/,Mapping={}]{SourceCodePro-ExtraLight}

% the name of file or directory
\newcommand\filename[1]{\texttt{#1}}

% the content of file
\newenvironment{file}%
{\verbatim}%
{\endverbatim}

% awk program, from package verbatim
\newenvironment{awkcode}%
{\verbatim}%
{\endverbatim}

% shell command, from package verbatim
\newenvironment{shell}%
{\verbatim}%
{\endverbatim}

% pattern for many situations
\newenvironment{pattern}%
{\begin{quotation}}%
{\end{quotation}}

% environment for summary
% TODO: 使用 mdframed 定义 summary
\newmdenv{summaryframe}
\newenvironment{summary}[1]
{
    \begin{summaryframe}
    \begin{center} \Large{#1} \end{center}%
}
{
    \end{summaryframe}
}

% term in English
\newcommand\term[1]{\textit{#1}}
% term in Chinese
\newcommand\cterm[1]{\textbf{#1}}

% subsection unnumbered
\CTEXsetup[number={}]{subsection}

% word that shows frequently
\newcommand\awk{\texttt{awk}}
\newcommand\print{\texttt{print}}
\newcommand\printf{\texttt{printf}}
\newcommand\nf{\texttt{NF}}
\newcommand\nr{\texttt{NR}}
\newcommand\AND{\texttt{\&\&}}
\newcommand\OR{\texttt{||}}
\newcommand\NOT{\texttt{!}}
\newcommand\BEGIN{\texttt{BEGIN}}
\newcommand\END{\texttt{END}}
\newcommand\length{\texttt{length}}
\newcommand\while{\texttt{while}}
\newcommand\for{\texttt{for}}
\newcommand\patact{\ \mbox{模式}\mbox{--}动作\ }
\newcommand\stmt{\textit{statements}}
\newcommand\expr{\textit{expression}}
\newcommand\regexpr{\textit{regular expression}}
\newcommand\pat{\textit{pattern}}
\newcommand\fs{\texttt{FS}}
\newcommand\OFS{\texttt{OFS}}
\newcommand\ctn{\texttt{continue}}
\newcommand\fmt{\textit{format}}

\theoremstyle{plain}
\theoremheaderfont{\bfseries}
\theorembodyfont{\normalfont}
\newtheorem{exercise}{Exercise}[chapter]
\newcommand\myexer{\textbf{Exercise\ }}

% 设置目录中 subsection 的缩进
\settowidth\cftsubsecindent{2em}
% 设置目录中 chapter 章节编号的宽度 (ctex 章节编号为中文, 需要特别注意).
% 参考 <<LaTeX 入门>>, 刘海洋 编著, 电子工业出版社, 2013.6
\settowidth\cftchapnumwidth{第十章} % 最宽的可能编号
\renewcommand\cftchapaftersnumb{\hspace{0.5em}} % 额外间距
\title{AWK 程序设计语言}
\author{Alfred V. Aho \and Brian W. Kernighan \and Peter J. Weinberger}